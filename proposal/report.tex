% A simple template for LaTeX documents
% 
% To produce pdf run:
%   $ pdflatex paper.tex 
%


\documentclass[12pt]{article}

% Begin paragraphs with new line
\usepackage{parskip}  

% Change margin size
\usepackage[margin=1in]{geometry}   

% Graphics Example:  (PDF's make for good plots)
\usepackage{graphicx}               
% \centerline{\includegraphics{figure.pdf}}

% subfigures, side by side
\usepackage{subcaption}

% hyperlinks
\usepackage{hyperref}

% Blocks of code
\usepackage{listings}
\lstset{basicstyle=\ttfamily, title=\lstname}
% Insert code like this. replace `plot.R` with file name.
% \lstinputlisting{plot.R}

% Monospaced fonts
%\usepackage{inconsolata}
% GNU \texttt{make} is a nice tool.

% Supports proof environment
\usepackage{amsthm}

% Allows writing \implies and align*
\usepackage{amsmath}

% Allows mathbb{R}
\usepackage{amsfonts}

% Numbers in scientific notation
% \usepackage{siunitx}

% Use tables generated by pandas
\usepackage{booktabs}

% Allows umlaut and non ascii characters
\usepackage[utf8]{inputenc}

% norm and infinity norm
\newcommand{\norm}[1]{\left\lVert#1\right\rVert}
\newcommand{\inorm}[1]{\left\lVert#1\right\rVert_\infty}

% Statistics essentials
\newcommand{\iid}{\text{ iid }}
\newcommand{\Exp}{\operatorname{E}}
\newcommand{\Var}{\operatorname{Var}}
\newcommand{\Cov}{\operatorname{Cov}}


%%%%%%%%%%%%%%%%%%%%%%%%%%%%%%%%%%%%%%%%%%%%%%%%%%%%%%%%%%%%

\begin{document}

\title{R Code Dependency Graphs}
\date{\today}
\author{Clark Fitzgerald}
\maketitle

\begin{abstract}

\end{abstract}

\section{Introduction}
%%%%%%%%%%%%%%%%%%%%%%%%%%%%%%%%%%%%%%%%%%%%%%%%%%%%%%%%%%%%

The idea behind code analysis is to treat the code itself as a data
structure as ``Programming on the language'' opens up rich possibilities.
This is an old idea, both the R and Julia language references cite Lisp
as the inspiration \cite{Rlang} \cite{bezanson2014julia}.

Compilers have used all manners of static analysis and
intermediate optimizations to create more efficient code. Interpreted
languages are much more limited in this respect. This project explores
the use of an alternative evaluation model to improve performance while
preserving language semantics.

The evaluation model for interpreted languages is simple. Each
expression of code is evaluated in the order that it appears. Informally
each expression is a line of code. This can be
viewed as a set of constraints on the evaluation order of the expressions:
\begin{enumerate}
    \item expression 1 executes before expression 2
    \item expression 2 executes before expression 3
    \item $\dots$
\end{enumerate}
What if these constraints are relaxed? Suppose expression 1 defines the variable
\texttt{x}, which is not used until expression 17. Then one has the
constraint:
\begin{enumerate}
    \item expression 1 executes before expression 17
\end{enumerate}
This can be generalized into a directed graph by considering expressions as
nodes and constraints as edges. The edges are implicit based on the order
of the statements in the code. Add an edge from $i \rightarrow k$ if
expression $k$ depends on the execution of expression $i$.  It's safe to
assume $i < k$, because expressions appearing later in a program can't
affect expressions which have already run. Hence the expression graph is
acyclic, ie. a DAG.

Scheduling execution based on the expression graph allows some expressions to execute in
parallel. For example, the following adjacent lines are independent, so
they can be computed simultaneously:

\begin{verbatim}
sx = sum(x)
sy = sum(y)
\end{verbatim}

Mathematically, the standard evaluation model is a total ordering on the
set of expressions in the code. The dependency graph is a partial ordering.

On a broader note, this is about embedding more intelligence into the 
system. Many languages have mechanisms or third party tools for explicitly
requesting asynchronous evaluation. These typically require changing the
code. This introduces complexity, makes maintenance more difficult, and
makes the code less portable.  It's more convenient to have one version of
the code which can be passed to a system which will just ``do the right
thing'', adjusting to different platforms, work loads, and data sizes on
the fly. 

\section{Literature Review}
%%%%%%%%%%%%%%%%%%%%%%%%%%%%%%%%%%%%%%%%%%%%%%%%%%%%%%%%%%%%

The expression graph proposed above is similar to 
use-definition and definition-use chains. A definition-use chain consists
of all expressions using a variable following the definition of that
variable. This amounts to a subset of the edges in the expression graph,
since it's possible that expressions depend on each other without
variables. For example, consider the following R code to save a plot to a
pdf:
\begin{verbatim}
pdf("xy.pdf")
plot(x, y)
title("x and y")
dev.off()
\end{verbatim}
These expressions depend on each other, but only \texttt{plot(x, y)} uses variables.

The Use-definition chain has been around since at least 1978
when it was used to remove dead (unused) code \cite{kennedy1978use}.
Code usefulness is defined recursively; a computation is useful if the result is
used later by another computation. This is combined with the ``base case''
of usefulness, a set of operations considered
intrinsically useful. This author considers calls to subroutines and branch
test instructions as intrinsically useful. We might take this idea and
tweak it a little- define expressions in a data analysis script as
intrinsically useful if they have a side effect, for example
saving data to disk.

More general than the use-definition chain is the program dependence graph (PDG)\cite{ferrante1987}:
\begin{quote}
    A PDG node represents
    an arbitrary sequential computation (e.g., a basic block, a
    statement, or an operation). An edge in a PDG represents
    a control dependence or a data dependence. PDGs do not
    contain any artificial sequencing constraints from the
    program text; they reveal the ideal parallelism in a
    program. \cite{sarkar1991automatic} 
\end{quote}
\cite{sarkar1991automatic} goes on to make the practical distinction between ideal
parallelism and useful parallisism. Overhead implies that the two often
differ.
The expression graph differs from the PDG since it allows the permutation of
operations in a basic block.

The hierarchical task graph (HTG) was introduced in \cite{girkar1992automatic}
to detect task parallelism in source code for use in compilers.
As in the others, it incorporates the control flow for a fine grained
parallelism. They allow `compound nodes' containing nested HTG's.
\cite{cosnard1995automatic} describe constructing a task graph based on
annotating the source code the program.
\cite{adve2004parallel} presents a model for predicting the run time of
programs based on a task graph. 

The literature cited in this section focuses primarily on compiled
languages along with careful analysis of control flow. Most examples and
applications presented along with these papers are for well-defined
algorithmic problems. These algorithmic problems are often quite different
than a high level data analysis script which may call down into several
different algorithms.

\section{Languages Requirements}
%%%%%%%%%%%%%%%%%%%%%%%%%%%%%%%%%%%%%%%%%%%%%%%%%%%%%%%%%%%%

We can consider building the expression graphs described above for languages that are
\begin{itemize}
    \item open source
    \item used for data analysis
    \item high level
    \item interpreted
    \item enable metaprogramming
\end{itemize}
Metaprogramming warrants more explanation. This refers to programmatically
inspecting and potentially modifying code from within the language.  It is
needed to determine when variables are created and used, among other
things.  The current popular languages satisfying these requirements are
Python, Julia, and R.

The basic unit to analyze is a single code expression.
Consider how an expression uses symbols, aka variables or names.
An expression may do any combination of the following things:
\begin{enumerate}
    \item Define new symbols: \texttt{x = 10}
    \item Use existing symbols: \texttt{sin(x)}
    \item Redefine existing symbols: \texttt{x = 20}
\end{enumerate}

\begin{table}[]
\centering
    \caption{Sorting \texttt{x} in place}
\label{tab-sort}
\begin{tabular}{ll}
    \textbf{language} & \textbf{code}        \\
\hline
    Python   & \texttt{x.sort()}    \\
    Julia    & \texttt{sort!(x)}    \\
    R        & \texttt{x = sort(x)}
\end{tabular}
\end{table}

Consider sorting a numeric vector as in Table \ref{tab-sort}.
The Julia and Python methods modify their arguments in place.  From a
computational standpoint this is great, since it allows the implementations
to use more space efficent sorting techniques. However, from a code
analysis standpoint this behavior is undesirable, since it means that we
need to assume generally that every method call in these languages both
uses and updates the object. Since data analysis scripts mainly consist of
function and method calls this will excessively contrain the problem.

It may be possible to recursively examine all functions and methods which
are used, but this is not ideal for a couple reasons. First, it would
require analyzing the underlying library code, which is orders of magnitude
more code than what the user has written. Second, eventually we'll get to
compiled code which requires totally different methods. For example, LLVM
can be used to programmatically generate use-definition chains
\cite{lattner2004llvm}.

Hence functional programming and pass by value semantics make constructing
graphs much more feasible. The R language is ideal in this respect.

\section{Related Work}
%%%%%%%%%%%%%%%%%%%%%%%%%%%%%%%%%%%%%%%%%%%%%%%%%%%%%%%%%%%%

Bengston's \texttt{future} package provides a mechanism for
asynchronous assignment and evaluation of R expressions \cite{R-future}. Once the
expression graph is created it might be possible to use similar mechanisms
for evaluation.

Jim Hester's covr package \cite{R-covr} checks unit test coverage of R
code. It is a practical example of computing on the language,
programmatically modifying the code by recording calls as they are made.
He pointed out a nice relevant idea distinguishing between AST's, parse
trees, and "lossless syntax trees":
\url{https://news.ycombinator.com/item?id=13628412}. To inject code into a
script one needs to be very
careful to preserve structure lost after parsing, ie.
comments and formatting. This is a non-trivial task.

A similar strategy of recording calls should work to collect timings and
resource usage for functions called with various data sizes. This could be
implemented with something like \texttt{System.time()}. This would allow us
to time each expression. Then we can potentially use this to change the
execution: ie. if $n > 1000$ then run a multicore version. This resembles
something like Profile Guided Optimization (PGO). 

Maybe a simpler way to do this is to just use the built in profiler. Use
the results to determine whether parallelization is worth it, and maybe set
some bounds for expected performance changes if one uses various forms of
parallelism. 

There may be potential to use statistical methods for this.
Run it many times, collecting profiling results for various values and use
this as the training data. After all, we have easy access to all the
machine learning type things that we need. We can use it to automatically
tune. But this implies that there is some way to tune it. Right now the
only "trick" I have up my sleeve is to try to parallelize it.

The vignette in Tierney's \texttt{proftools} package has some nice examples
of visualizing profiling data \cite{R-proftools}. The call graphs and related visualizations
are conceptually similar to what might be done with the expression graph.
\texttt{profvis} integrates with the IDE to indicate the actual
line of source code along with the related timing info \cite{R-profvis}.

Xie's \texttt{knitr} facilitates reproducible computations for
chunks of code in Rmarkdown documents \cite{R-knitr}. One feature it enables is caching,
ie. it doesn't need to run a chunk of code if nothing has changed. One can
manually specify the chunk dependencies by relative or absolute indices,
ie. -2 for the chunk 2 blocks in front of the current chunk, or 1 for the
first chunk. This seems unreliable because it requires the user to
accurately infer the dependency information, and it doesn't automatically
adjust if one inserts new chunks in the document.

Knitr also has an \texttt{autodep} option to infer this dependency
information. This works by comparing the global variables existing before
and after running the code in each chunk. It stores this information in
special files. So it doesn't use any static analysis of the code.

But the structure of knitr blocks here is actually very appealing- this is
a great use case for the parallelism. Why not evaluate the chunks in
parallel if possible? I wonder how difficult it would be to hook into
knitr's system... And while I'm at it, why not Jupyter Notebooks? If I can
figure out how to build an equivalent code dependency graph for Python that
is.

\section{Graph Construction}
%%%%%%%%%%%%%%%%%%%%%%%%%%%%%%%%%%%%%%%%%%%%%%%%%%%%%%%%%%%%

To start off we make two assumptions on the program. First it's assumed to
be correct, meaning that it will run sequentially without errors.
Second every expression should be strictly necessary. This can be achieved
through a preprocessing step removing dead code.

The expression graph and related data structures are related to the parsed
script, referred to as the \textbf{parse tree}. The expression
graph is a function of the parse tree. Since the parser
doesn't care about non significant white space and comments,  different
scripts can produce the same parse tree.  Many parse trees can give rise to
the same code graph. For example, the expression graph shouldn't care if one uses
\texttt{=} or \texttt{<-} for assignment. Nor will it care about the
ordering of two adjacent lines binding a symbol to a literal constant:

\begin{verbatim}
a = 1
b = 2
\end{verbatim}

Therefore we lose information by converting from a parse tree to an
expression graph.

We can also introduce an artificial (graph style) source and sink representing the
beginning and end of the program, respectively.

Figures \ref{fig:ast} and \ref{fig:codegraph} illustrate the ideas of
a parse tree an expression dependency
graph for the four lines of code in listing \ref{list:ab}.  Edges 1 and 3
represent the respective uses of the variable \texttt{n} and \texttt{x}.
Edge 2 comes from the redefinition of \texttt{n}.  Edge 5 propagates the
most recent definition of \texttt{n}.  The least obvious is edge 4, which
is necessary to respect R's lexical scoping semantics since \texttt{x <-
rnorm(n)} uses the first definition of \texttt{n}. The general rule here is
that all statements using one version of \texttt{n} must execute before
\texttt{n} can be redefined.

The dashed edges in \ref{fig:codegraph} are redundant for representing
expression dependence given the other edges. Indeed, the code in
listing \ref{list:ab} must run sequentially. One may wish to remove such
redundant edges, especially for visual presentation of a larger program.

\lstinputlisting[language=R, caption=Simple script, label=list:ab]{../experiments/ast/ab.R}

\begin{figure}
\centering
\begin{subfigure}{.6\textwidth}
    \centering
    \includegraphics[width=.8\linewidth]{../experiments/ast/ast.pdf}
    \caption{Parse tree}
    \label{fig:ast}
\end{subfigure}%
\begin{subfigure}{.4\textwidth}
  \centering
  \includegraphics[width=.8\linewidth]{../experiments/ast/codegraph.pdf}
  \caption{Expression dependency graph}
  \label{fig:codegraph}
\end{subfigure}
\caption{Different representations of the script in Listing \ref{list:ab}}
%\label{fig:test}
\end{figure}

The existence of some edges may depend on conditional statements which can't
be known until run time. In this case the conservative and correct way to
handle the situation is to add the edges in question. For example, in the
following code one assumes that the expression \texttt{x <- 10} will run.

\begin{verbatim}
# coinflip() randomly returns TRUE or FALSE
if(coinflip()){
    x <- 10
}
\end{verbatim}

\section{Task Based Parallelism}
%%%%%%%%%%%%%%%%%%%%%%%%%%%%%%%%%%%%%%%%%%%%%%%%%%%%%%%%%%%%

It's reasonable to try to improve code performance if slow speed affects
many users. For scripts, superficially it might appear that few people are
affected. Indeed, if a researcher writes one script that takes a couple
minutes to run, and they run it a couple times then it doesn't matter much,
and there's no point in attempting to accelerate it.  However, these same
sorts of "scripts" can be used in more serious ways.  For example, scripts
can be used as Extract Transform Load (ETL) tools that run as batch jobs
every day or every hour. Then the script runs many times, so it's important
to realize more performance. For ease of use and maintainability it's nice
to have automatic tools to accelerate the performance. One can write
natural, beginner level scripts and have them run much faster.

As of 2009, most efforts to parallelize R have focused on the lower level
programming mechanisms \cite{schmidberger2009state}. These needed to be in
place before any higher level automatic detection could be built and
function.

Let $k$ be the number of cores on a machine.
To accelerate code using this single machine the best possible case is if
we can keep all $k$ cores busy at once. This will happen if there are $nk$
expressions which can be independently scheduled for some positive integer
$n$. 
On a two core machine that script might look something like:

\begin{verbatim}
# These could be run in parallel
a = long_running_func()
b = long_running_func()
\end{verbatim}

The worst possible case is if the second long running computation depends
on the first, and everything else depends on the second. Then it must run
in serial so parallelism can't help.

\begin{verbatim}
a = long_running_func()
b = long_running_func2(a)  # depends on a
# Now perform many operations on b
\end{verbatim}

\subsection{Static Execution}

If overhead and expression run time are approximately known then the
question of optimal execution for the complete expression dependency graph
can be framed as a scheduling optimization problem and solved statically.
The objective function to minimize is the total wall clock time to complete
execution. Constraints come from the depenedency graph and that at most $k$
cores may be active at one time.

To consider an alternative evaluation model the overhead required to
parallelize an expression should require less time than running the
expression itself. Rounding to orders of magnitude, here are some rough
times for reference executing on a modest machine. Simple R expressions
take $10^{-7}$ seconds to evaluate. Using a system level parallel fork
requires $10^{-3}$ seconds of overhead. Evaluation on an existing local
socket cluster takes $10^{-4}$ seconds. Then either of these well
established methods for parallelism in R won't become efficient until the
code under evaluation takes on the order of $10^{-3}$ seconds.
These timings include latency for interprocess communication on a single
machine. Bandwidth is also an issue, since serializing large amounts of
data between processes, ie. millions of floating point numbers, will impact
performance. For example, listing \ref{list:overhead} squares each element
of a vector of one million floating point numbers. Memory transfer overhead causes
a parallel evaluation to take more than an order of magnitude more time
than the regular version. 
Technical solutions such as threading and shared memory have the
potential to reduce these sources of overhead.

\lstinputlisting[language=R, caption=Overhead caused by memory transfer,
label=list:overhead]{overhead.R}

Issues of latency and bandwidth generally become more complex for different
architectures such as distributed systems and GPU's
\cite{matloff2015parallel}.

\subsection{Dynamic Execution}

Alternatively a dynamic execution model can be used. This relies on a
master / worker architecture. The reference version could be a multicore
system which forks to evaluate expressions.  At a high level, the master
runs an event loop that pops expressions from the top of the expression
dependency graph.
This approach is appealing
because it can do dynamic load balancing without needing to know.

Here's an algorithm: Insert the artifical node 0 representing the beginning
of the script, so that nodes without parents now have node 0 as a parent. Mark
these direct descendants of node 0 as ready. Let the workers begin evaluating
these expressions.

\begin{enumerate}
    \item Event loop checks to see if any are done.
    \item Expression $e_i$ finishes and the results are available again
        on master.
    \item For each expression $e_j$ which depends directly on $e_i$: check
        if $e_j$ has no other existing parents then mark it as ready.
    \item Remove $e_i$ from the graph.
    \item Free workers begin executing any of the nodes that are marked as ready.
\end{enumerate}

There's some nuance here by trying to keep each of $k$ workers as busy as
possible while still respecting the constraints. Ie. there may be
bottlenecks where only one worker can be active, but after that all the
others can go.

This algorithm could also be refined into a priority queue by keeping the ready
nodes in a heap, with the values determining the heap order as the number
of expressions that depend on that expression, directly or indirectly. This
is a little naive- it would be better to have timings of the code and do
something more optimal in terms of reducing run time.

The process forking every time will be quite inefficient. The more intelligent
thing to do is `pipeline' the operations, and send whole related blocks of
expressions to various processes to evaluate.

\section{How CodeDepends works}
%%%%%%%%%%%%%%%%%%%%%%%%%%%%%%%%%%%%%%%%%%%%%%%%%%%%%%%%%%%%

CodeDepends is the underlying package which generates the expression
dependency information \cite{R-CodeDepends}.

The parsing uses an object oriented wrapper around R's builtin \texttt{parse()}
which handles different file types ie. script or dynamic documents among other arguments.
This doesn't directly expose code comments.
At first glance it seems CodeDepends uses the tokens
more indirectly, through functions like \texttt{is.name()}. 

The workhorse functions are in \texttt{CodeDepends.R}. Overall, the
approach resembles \texttt{codetools::walkCode} as discussed in
\cite{chambers2016extending}. Essentially it walks the tree of code, calling
a function to collect usage information.

When analyzing a single expression, first a 
collector object is created with \texttt{inputCollector()}. This is a
closure that maintains a list of everything in the expression that has been seen
so far: files, variables, function calls, etc. It returns a list of
functions to update the data in the closure.

\texttt{getInputs.language()} takes a collector object and
recurses through expressions until it finds the leaf nodes which can be functions, calls, assignments,
names, literals, or pairlists. Upon finding one of these leaf nodes it
calls the collector object. 
Many special cases of functions are handled in functionHandlers.R, such as
\texttt{\$, rm, for} as well as non standard evaluation. Special attention seems
to have been paid to dplyr operations.

\section{Challenges}
%%%%%%%%%%%%%%%%%%%%%%%%%%%%%%%%%%%%%%%%%%%%%%%%%%%%%%%%%%%%

\subsection{Non Standard Evaluation}

`fit = lm(y ~ x)` doesn't detect the dependency of `fit` on `x, y`.

Following the control flow for `lm` we see the following calls:
`model.matrix.lm`, `model.frame.default`, `as.formula`. But eventually this
`y ~ x` itself must be evaluated in which case `.Primitive("~")` is called,
which certainly does something weird. But what?

Haven't yet checked things like:
`<<- assign`

Recursion? Iterating updates?


\subsection{Reproducibility}

Reproducing random streams allows one to perform the exact same random
computation. This is useful for investigating ano The R documentation for
\texttt{parallel::mcparallel} explains how this works in the case of
parallel forking:

\begin{quote}
     The behaviour with ‘mc.set.seed = TRUE’ is different only if
     ‘RNGkind("L'Ecuyer-CMRG")’ has been selected.  Then each time a child
     is forked it is given the next stream (see ‘nextRNGStream’).  So if
     you select that generator, set a seed and call ‘mc.reset.stream’ just
     before the first use of ‘mcparallel’ the results of simulations will
     be reproducible provided the same tasks are given to the first,
     second, ...  forked process.
\end{quote}

\subsection{Dynamic Evaluation}

Correct, legitimate code can be written that depends on the results of
dynamic evaluation. Indeed, some symbols may not currently exist. This is
more common inside package code that defines and uses many functions, and
less common in scripts.  Here's an example:

\begin{verbatim}
f = function() 0        # 1
g = function() f() + 1  # 2
f = function() 10       # 3
g()                     # 4
\end{verbatim}

The last line returns 11, since it uses most recent version of \texttt{f()}.
An expression dependency graph that does not correctly handle the dynamic
evaluation here will impose this partial order:

\begin{verbatim}
1 -> 2, 1 -> 3, 2 -> 4
\end{verbatim}

So the statements could be written in the following order, which respects
the partial order:

\begin{verbatim}
f = function() 0        # 1
g = function() f() + 1  # 2
g()                     # 4
f = function() 10       # 3
\end{verbatim}

In this case the call to \texttt{g()} will incorrectly return 1 instead of 11.
Hence there is a ``hidden'' dependency implicit here: \texttt{3 -> 4}.
This comes back to lexical scoping rules.

One possible way to get around this is to first inline all user defined
functions, and then perform the expression dependency analysis. One
substitutes the bodies of \texttt{g(), f()} so the code presented above
becomes simply \texttt{10 + 1}. This also has the effect of removing user
defined functions from the expression graph.

\subsection{Mutable Objects}

Environments and reference class objects are mutable in R. 

\section{Scripts in the Wild}
%%%%%%%%%%%%%%%%%%%%%%%%%%%%%%%%%%%%%%%%%%%%%%%%%%%%%%%%%%%%

I could do some analysis on scripts that I scrape to determine theoretical
bounds for the number of statements that must be run sequentially,
and the max number of concurrent workers. My example script I've been
working on has about 50 statements, and I think min number of sequential
statements is around 10. Need to check. This could motivate reducing
overhead though.

Duncan suggests to find R code that's unacceptably slow, and see what these
methods can do to improve it. So to get a better sense of theoretical
improvements I could execute the code in normal R and attach timings for
each expression. This would show me true lower bounds for the method. It
would also be quite interesting to have some sense of the distribution of
execution times for different types of code. I imagine it varies
dramatically, but there may be rough patterns. One result I'm expecting is
that just one or two lines dominate the total script run time. In which
case anything I do can't really help. Unless that line calls one function
and I can use this method on the one function.

Vignettes, including those found on bioconductor, are for instructional
purposes, so they're mostly simple and self contained. This isn't the use
case I'm targeting. They are nice because the data should be included with
the package, thus I have some hope of actually being able to run them if
needed.

To get some idea of the potential for speedup, if I have an unlimited
number of cores, no overhead, and each expression takes roughly the same
amount of time to execute then the script can run no faster than the length
of the longest path in the graph. Hence speedup will be the total number
of statements divided by the length of the longest path in the graph.
By the way, all of those assumptions are unreasonable.


\appendix
\section{Definitions}
%%%%%%%%%%%%%%%%%%%%%%%%%%%%%%%%%%%%%%%%%%%%%%%%%%%%%%%%%%%%

We need to start out with some definitions and basic concepts to make all
of this precise. The following material is from the R
Language Reference \cite{Rlang}.

``A \textbf{statement} is a
syntactically correct collection of tokens.'' Less formally one can think of
a statement as a single line of code. The `line' rule doesn't always hold in practice,
since a semicolon can put two statements on one line, and a single
statement may span multiple lines.

\begin{verbatim}
# Two statements on line
a = 1; b = 2

# One statement on multiple lines
plot(x,
     y)
\end{verbatim}

\textbf{symbol} is a variable name such as \textbf{a, b} above.  The words
symbol, variable, and name are used interchangeably. For consistency I'll
stick with symbol.

Assignment is the binding of a symbol to an R object.

``An \textbf{expression} contains one or more statements.'' Expression objects
in the langauge contain parsed but unevaluated statements. 

Statements can be grouped together using braces to form a \textbf{block}.
Since expressions can be nested, we can consider a block just a special
type of expression.

\begin{verbatim}
{
a = 1
b = 2
}
\end{verbatim}

\section{LLVM}
%%%%%%%%%%%%%%%%%%%%%%%%%%%%%%%%%%%%%%%%%%%%%%%%%%%%%%%%%%%%

LLVM has some functionality for working with these use / def chains
\cite{Lattner2004}. For example:
\url{http://llvm.org/docs/ProgrammersManual.html#iterating-over-def-use-use-def-chains}.
Following their terminology, an object of class \texttt{Value} is used by
potentially many objects of class \texttt{User}. The \emph{def-use} chain
is the list of all \texttt{Users} for a particular \texttt{Value}. I think
of this as all the places in the code where this variable propagates. In
contrast, the \emph{use-def} chain is the list of all \texttt{Values} for a
particular \texttt{User}. This is all the inputs to a newly
created object. There's room for both of these chains to be expanded recursively.

In this example the def-use chain for \texttt{x} is only \texttt{[y]}, but the
recursive one is \texttt{[y, z, z2]}. The use-def chain for \texttt{y} is
\texttt{[z, z2]}.

\begin{verbatim}
x = 10
y = x + 5
z = y + 2
z2 = y + 100
\end{verbatim}

We might be able to use this along with R code to determine if an R
function calling C code is pure.


\bibliographystyle{plain}
\bibliography{citations} 

\end{document}
