% A simple template for LaTeX documents
% 
% To produce pdf run:
%   $ pdflatex paper.tex 
%

\documentclass[12pt]{article}

% Begin paragraphs with new line
\usepackage{parskip}  

% Change margin size
\usepackage[margin=1in]{geometry}   

% Graphics Example:  (PDF's make for good plots)
\usepackage{graphicx}               
% \centerline{\includegraphics{figure.pdf}}

% Allows hyperlinks
\usepackage{hyperref}

% Blocks of code
\usepackage{listings}
\lstset{basicstyle=\ttfamily, title=\lstname}
% Insert code like this. replace `plot.R` with file name.
% \lstinputlisting{plot.R}

% Monospaced fonts
%\usepackage{inconsolata}
% GNU \texttt{make} is a nice tool.

% Supports proof environment
\usepackage{amsthm}

% Allows writing \implies and align*
\usepackage{amsmath}

% Allows mathbb{R}
\usepackage{amsfonts}

% Numbers in scientific notation
% \usepackage{siunitx}

% Use tables generated by pandas
\usepackage{booktabs}

% norm and infinity norm
\newcommand{\norm}[1]{\left\lVert#1\right\rVert}
\newcommand{\inorm}[1]{\left\lVert#1\right\rVert_\infty}

% Statistics essentials
\newcommand{\iid}{\text{ iid }}
\newcommand{\Exp}{\operatorname{E}}
\newcommand{\Var}{\operatorname{Var}}
\newcommand{\Cov}{\operatorname{Cov}}


%%%%%%%%%%%%%%%%%%%%%%%%%%%%%%%%%%%%%%%%%%%%%%%%%%%%%%%%%%%%

\begin{document}

\title{Fall 2016 report}
\date{December 9, 2016}
\author{Clark Fitzgerald}
\maketitle

This is the final report for Professor Duncan Temple Lang for 8 units of
credit in Fall 2016.  The tentative plan is to take the qualifying exam in
Spring 2017. I'll focus on code analysis as the primary topic. Code
analysis will be used as a base for further work, ie. detecting and using
parallelism.

\begin{emph}
Describe 3 or 4 things to focus on, why they're interesting, what are the challenges,
and who cares?
\end{emph}


\section{Code Generation}
%%%%%%%%%%%%%%%%%%%%%%%%%%%%%%%%%%%%%%%%%%%%%%%%%%%%%%%%%%%%

This would be an extension of Duncan's work with the RCIndex \cite{R-RCIndex} and
RCodegen \cite{R-RCodegen} packages, along with other prior work.


The first task here is to create software that generates R bindings for an
existing C and C++ libraries. 

This is useful because writing these bindings by hand
has several problems: 
Writing by hand is error prone.
It's tedious to write 1000's of wrappers.
Hard to update wrapping package if software version is different.

This is interesting because it allows one to very quickly add new
capabilities to R as a system. The new capabilities then facilitate rapid
prototyping and novel types of data analysis, using R as a "glue language".

Duncan describes further applications and motivations in the paper in
the RCIndex pacakge repository \cite{R-RCIndex}.

The approach in RCIndex is different from existing software like SWIG
\cite{swig} and Rcpp \cite{R-Rcpp} because it hooks directly into the clang
compiler rather than processing the text itself. This means we have a code
analysis tool that will be more robust and consistent with behavior of a compiler. 

\subsection{Challenges}

Garbage collection is potentially difficult. A C routine might allocate
memory that R knows nothing about. Then how is that memory freed? A
possible solution to this problem is to look in the body of the code
itself, which RCIndex doesn't yet do.



\subsection{Applications}

The mature C++ computer vision library openCV
\cite{opencv_library} is an example of capabilities that we would like to
access from within R. It's possible to directly write computer vision
algorithms in R, but this is a huge amount of duplicated work if a mature
library already exists.

Another application is cutting edge specialized machine learning code like
Hogwild++ \cite{zhang2016hogwild}.


\section{Transportation Data}
%%%%%%%%%%%%%%%%%%%%%%%%%%%%%%%%%%%%%%%%%%%%%%%%%%%%%%%%%%%%




\bibliography{citations} 

\bibliographystyle{apalike}

\end{document}
