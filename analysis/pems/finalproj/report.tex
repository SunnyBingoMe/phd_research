% A simple template for LaTeX documents
% 
% To produce pdf run:
%   $ pdflatex paper.tex 
%

\documentclass[12pt]{article}

% Begin paragraphs with new line
\usepackage{parskip}  

% Change margin size
\usepackage[margin=1in]{geometry}   

% Graphics Example:  (PDF's make for good plots)
\usepackage{graphicx}               
% \centerline{\includegraphics{figure.pdf}}

% Allows hyperlinks
\usepackage{hyperref}

% Blocks of code
\usepackage{listings}
\lstset{basicstyle=\ttfamily, title=\lstname}
% Insert code like this. replace `plot.R` with file name.
% \lstinputlisting{plot.R}

% Monospaced fonts
%\usepackage{inconsolata}
% GNU \texttt{make} is a nice tool.

% Supports proof environment
\usepackage{amsthm}

% Allows writing \implies and align*
\usepackage{amsmath}

% Allows mathbb{R}
\usepackage{amsfonts}

% Numbers in scientific notation
% \usepackage{siunitx}

% Use tables generated by pandas
\usepackage{booktabs}

% norm and infinity norm
\newcommand{\norm}[1]{\left\lVert#1\right\rVert}
\newcommand{\inorm}[1]{\left\lVert#1\right\rVert_\infty}

% Statistics essentials
\newcommand{\iid}{\text{ iid }}
\newcommand{\Exp}{\operatorname{E}}
\newcommand{\Var}{\operatorname{Var}}
\newcommand{\Cov}{\operatorname{Cov}}


%%%%%%%%%%%%%%%%%%%%%%%%%%%%%%%%%%%%%%%%%%%%%%%%%%%%%%%%%%%%

\begin{document}

\title{ECI 256 Final Project}
\date{December 2016}
\author{Clark Fitzgerald}
\maketitle

\section{Abstract}

For this project we 

The goal of this project is to search for the presence of congestion
shockwaves in the raw 2016 PeMS data.

\section{Review}

\cite{lu2007freeway} describe a numerical algorithm to detect shockwaves
based on trajectory data. Their vehicle trajectory plots appear to show
shockwaves that last for 20 to 30 seconds for a fixed location.

Given sufficiently dense velocity observations over space and time it may
be possible to recover "average" trajectories. In the case of the PeMS data
we have the mean velocity for 30 second intervals. Then a shockwave lasting
20-30 seconds would manifest as a lower mean velocity for one or two
periods.

\cite{williams2003modeling} discuss using ARIMA models for univariate
traffic streams.

\quote{
The  ARIMA  ( 1,0,1 ) ( 0,1,1 ) model provides a simple three-parameter
linear recursive estima-tor. Therefore,  an  appropriately  applied
seasonal ARIMA modelshould   be   considered   the   parametric   model
benchmark   forunivariate  traffic  condition  forecasting.  
}

So we might ask the question: how different are each of these sensors in
terms of model parameters? We can fit seasonal trends and ARIMA
models to each of the stations. All the papers I looked at talk about
predicting traffic conditions.

Do they cluster into a few groups?
Areas with few onramps should be very different than dense urban areas.

Another thing to do- find all the active fixed bottlenecks in the 2016 PeMS data.
One general heuristic for this is to find all places that the velocity $v$ or
density $\rho$ is much lower upstream than downstream for at least some
period of time $T_{\min}$. Then it's reasonable to say there is a
bottleneck between those two mile markers.

We can do this with a two stage approach. First look at every timestamp and
check if the average velocity for station $i$ is below some threshold
$v_{i}$ while station $i + 1$, the next one downstream on the freeway has
traffic moving at velocity greater than $v_{i + 1}$. If this condition persists for
longer than $t_{\min}$ then call it a bottleneck between station $i$ and $i
+ 1$. This can then be used to check dimensions of the bottleneck in time
and space. There we go- that's a great title- Spatial temporal dimensions
of traffic bottlenecks.

\bibliography{citations} 

\bibliographystyle{apalike}

\end{document}
